\documentclass[a4paper]{report}
\usepackage[utf8]{inputenc}
\usepackage[portuguese]{babel}
\usepackage{hyperref}
\usepackage{a4wide}
\hypersetup{pdftitle={CG - Fase 4 - Grupo 7},
pdfauthor={João Teixeira, José Ferreira, Miguel Solino},
colorlinks=true,
urlcolor=blue,
linkcolor=black}
\usepackage{subcaption}
\usepackage[cache=false]{minted}
\usepackage{listings}
\usepackage{booktabs}
\usepackage{multirow}
\usepackage{appendix}
\usepackage{tikz}
\usepackage{authblk}
\usepackage{bashful}
\usepackage{verbatim}
\usepackage{amsmath}
\usepackage{tikz}
\usepackage{tikz,fullpage}
\usepackage{pgfgantt}
\usepackage{amssymb}
\usepackage{multirow}
\usepackage{mwe}
\usetikzlibrary{arrows,%
                petri,%
                topaths}%
\usepackage{tkz-berge}
\usetikzlibrary{positioning,automata,decorations.markings}
\AfterEndEnvironment{figure}{\noindent\ignorespaces}
\AfterEndEnvironment{table}{\noindent\ignorespaces}

\definecolor{solarized@base03}{HTML}{002B36}
\definecolor{solarized@base02}{HTML}{073642}
\definecolor{solarized@base01}{HTML}{586e75}
\definecolor{solarized@base00}{HTML}{657b83}
\definecolor{solarized@base0}{HTML}{839496}
\definecolor{solarized@base1}{HTML}{93a1a1}
\definecolor{solarized@base2}{HTML}{EEE8D5}
\definecolor{solarized@base3}{HTML}{FDF6E3}
\definecolor{solarized@yellow}{HTML}{B58900}
\definecolor{solarized@orange}{HTML}{CB4B16}
\definecolor{solarized@red}{HTML}{DC322F}
\definecolor{solarized@magenta}{HTML}{D33682}
\definecolor{solarized@violet}{HTML}{6C71C4}
\definecolor{solarized@blue}{HTML}{268BD2}
\definecolor{solarized@cyan}{HTML}{2AA198}
\definecolor{solarized@green}{HTML}{859900}

\lstset{
  language=Java,
  upquote=true,
  columns=fixed,
  tabsize=4,
  extendedchars=true,
  breaklines=true,
  numbers=left,
  numbersep=5pt,
  rulesepcolor=\color{solarized@base03},
  numberstyle=\tiny\color{solarized@base01},
  basicstyle=\footnotesize\ttfamily,
  keywordstyle=\color{solarized@green},
  stringstyle=\color{solarized@yellow}\ttfamily,
  identifierstyle=\color{solarized@blue},
  commentstyle=\color{solarized@base01},
  emphstyle=\color{solarized@red}
}

\begin{document}

\title{Computação Gráfica\\
\large Fase 4 - Grupo 7}
\author{José Ferreira (A83683) \and João Teixeira (A85504) \and Miguel Solino (A86435)}
\date{\today}

\begin{center}
    \begin{minipage}{0.75\linewidth}
        \centering
        \includegraphics[width=0.4\textwidth]{images/eng.jpeg}\par\vspace{1cm}
        \vspace{1.5cm}
        \href{https://www.uminho.pt/PT}
        {\color{black}{\scshape\LARGE Universidade do Minho}} \par
        \vspace{1cm}
        \href{https://www.di.uminho.pt/}
        {\color{black}{\scshape\Large Departamento de Informática}} \par
        \vspace{1.5cm}
        \maketitle
    \end{minipage}
\end{center}

\tableofcontents

\chapter{Introdução}
Este trabalho foi desenvolvido no âmbito da unidade curricular de computação
gráfica e está dividido em várias fases de entrega distintas sendo que esta é a
terceira.\\
O projeto foi dividido em 4 entragas faseadas sendo que esta é a última. Ao
longo deste relatório iremos descrever a metodologia e as soluções criadas para
solucionar todos os critérios propostos assim como descrever as funcionalidades
extra implementadas.\\
Primeiro adicionamos a todas as primitivas desenvolvidas anteriormente as
coordenadas de textura e as normais. Em seguida modificamos o engine para
conseguir ler os ficheiros gerados.\\
Em seguida adicionamos a possibilidade de definir as propriedades de um obejto
no \textit{XML}. Adicionamos ainda a possibilidade de definir luzes no
\textit{XML}.\\
Finalmente criamos mais \textit{scenes} e melhoramos as já existentes de forma a
fazer uso das novas funcionalidades do \textit{engine}.\\

\chapter{Generator}
Para facilitar o cálculo dos novos pontos foi criada uma nova classe chamada
\textit{ModelPoint}. Esta contém um \textit{Point} que representa as
coordenadas do ponto no espaço, um \textit{Vector} que representa a normal do
ponto e doi floats que representam as coordenadas da textura. Como métodos
apenas foram definidos um conjunto de construtores que recebem os vários tipos
de pontos e vetores e um \textit{toString} para facilitar a escrita dos pontos
para um ficheiro.\\
Assim, todas as primitivas desenvolvidas anteriormente foram alteradas:
\begin{itemize}
        \item Plano
        \item Esfera
        \item Torus
        \item Cilindro
        \item Cone
        \item Caixa
        \item Patches de Bezier
\end{itemize}
Todos os exmeplos apresentados ao longo deste capítulo têm o seu \textit{XML} na
pasta \textit{scenes/report} com o nome [\textit{PRIMITIVA}]\textit{.xml}.

\section{Plano}
Para desenhar um plano com dois triângulos sabendo o lado do plano basta
conseguir obter as coordenadas dos cantos do plano.\\
A normal do plano é sempre a mesma em todos os pontos e aponta para cima.\\
As coordenadas da textura são mapeadas diretamente.\\
\begin{figure}[H]
    \centering
    \begin{minipage}{0.33\textwidth}
        \centering
        \includegraphics[width=\textwidth]{images/esquema_plano.png}
        \caption{coord dos pontos}
    \end{minipage}\hfill
    \begin{minipage}{0.33\textwidth}
        \centering
        \includegraphics[width=\textwidth]{images/esquema_plano.png}
        \caption{coord das normais}
    \end{minipage}\hfill
    \begin{minipage}{0.33\textwidth}
        \centering
        \includegraphics[width=\textwidth]{images/esquema_plano.png}
        \caption{coord das texturas}
    \end{minipage}\hfill
\end{figure}

\begin{figure}[H]
    \centering
    \begin{minipage}{0.49\textwidth}
        \centering
        \includegraphics[width=\textwidth]{images/persian_carpet.jpg}
        \caption{exemplo de textura}
    \end{minipage}\hfill
    \begin{minipage}{0.5\textwidth}
        \centering
        \includegraphics[width=\textwidth]{images/persian_carpet_rendered.png}
        \caption{resultado da textura}
    \end{minipage}\hfill
\end{figure}

\section{Esfera}
Para desenhar a esfera é necessário saber o seu raio (\textit{\_radius}), o
número de slices (\textit{\_slices}) e o número de stacks (\textit{\_stacks}).\\
Fazendo uso do sistema de coordenadas esférica basta calcular qual é o azimute e
qual é a inclinação de cada ponto (visto que o raio do pontos é igual ao raio da
esfera).\\
A normal de uma esfera num dado ponto é igual ao vetor normalizado do vetor que
vai do centro da esfera a um dado ponto. Para agilizar esta conta foi criado um
método na classe \textit{Point} que realiza estas contas
(\textit{normalized\_vector}).\\
Para facilitar estas contas são calculados dois valores, o angulo entre cada
slice (\textit{a\_slice}) e o angulo entre cada stack (\textit{a\_stack}).\\
As coordenadas de textura de uma esfera são semelhantes às de um mapa, sendo que
para calcular as coordenadas de textura de uma esfera basta ver em que slice e
em que stack está um dado ponto e dividir esses valores pelo número total de
slices e de stacks respetivamente.

\begin{lstlisting}
float curr_t_x = slice / _slices;
float curr_t_y = stack / _stacks;
float next_t_x = (slice + 1) / _slices;
float next_t_y = (stack + 1) / _stacks;
\end{lstlisting}

\begin{figure}[H]
    \centering
    \begin{minipage}{0.33\textwidth}
        \centering
        \includegraphics[width=\textwidth]{images/sphere_vetores.png}
        \caption{coord dos pontos}
    \end{minipage}\hfill
    \begin{minipage}{0.33\textwidth}
        \centering
        \includegraphics[width=\textwidth]{images/sphere_vetores.png}
        \caption{coord das normais}
    \end{minipage}\hfill
    \begin{minipage}{0.33\textwidth}
        \centering
        \includegraphics[width=\textwidth]{images/sphere_vetores.png}
        \caption{coord das texturas}
    \end{minipage}\hfill
\end{figure}
Para guardar os pontos calculados é criado um vetor de \textit{ModelPoint}
chamado \textit{coords}.

\begin{lstlisting}
    for (i32 slice = 0; slice < _slices; slice++) {
        for (i32 stack = 0; stack < _stacks; stack++) {
            auto p0 = PointSpherical(_radius, a_stack * stack, a_slice * slice);
            auto p1 = p0.add_inclination(a_stack);
            auto p2 = p0.add_azimuth(a_slice);
            auto p3 = p2.add_inclination(a_stack);

            auto v0 = p0.normalized_vector();
            auto v1 = p1.normalized_vector();
            auto v2 = p2.normalized_vector();
            auto v3 = p3.normalized_vector();

            if (stack != 0) {
                // 1st triangle
                coords.emplace_back(p2, v2, next_t_x, curr_t_y);
                coords.emplace_back(p0, v0, curr_t_x, curr_t_y);
                coords.emplace_back(p3, v3, next_t_x, next_t_y);
            }
            if (stack != _stacks - 1) {
                // 2nd triangle
                coords.emplace_back(p3, v3, next_t_x, next_t_y);
                coords.emplace_back(p0, v0, curr_t_x, curr_t_y);
                coords.emplace_back(p1, v1, curr_t_x, next_t_y);
            }
        }
    }
\end{lstlisting}


\begin{figure}[H]
    \centering
    \begin{minipage}{0.60\textwidth}
        \centering
        \includegraphics[width=\textwidth]{images/guy.jpg}
        \caption{exemplo de textura}
    \end{minipage}\hfill
    \begin{minipage}{0.39\textwidth}
        \centering
        \includegraphics[width=\textwidth]{images/guy_rendered.png}
        \caption{resultado da textura}
    \end{minipage}\hfill
\end{figure}

\section{Torus}
Para desenhar uma Torus é preciso saber o raio interior de uma torus, o raio
exterior, o número de stacks(\textit{\_stacks} e o número de
slices(\textit{\_slices}).\\
Para facilitar os cálculos estes valores são internamente convertidos para o
raio que vai do centro da torus ao centro do anel da torus (\textit{\_radius}) e
o raio do anel propriamente dito (\textit{\_ring\_radius}).

\begin{figure}[H]
    \centering 
    \includegraphics[width=0.5\textwidth]{images/torus_points.png}
    \caption{Esquema dos pontos no plano}
    \label{fig:scheme_torus}
\end{figure}
Primeiro é definido o ângulo entre cada slice e cada stack com:
\begin{lstlisting}
float a_slice = 2.0f * M_PI / _slices;
float a_stack = 2.0f * M_PI / _stacks;
\end{lstlisting}
Para calcular as coordenadas de um ponto central da torus usamos coordenadas
esféricas. O raio é igual ao \textit{\_radius} e a inclinação é igual a $\pi$/2.
Logo, o único valor que varia é o azimute, que é igual a \textit{a\_slice} vezes
o índice da slice do ponto.\\
Com base nisto conseguimos calcular a coordenadas do ponto central de qualquer
\textit{slice}. Para calcular os triangulos do torus vai ser necessário ter o
ponto \textit{center} (que está na slice i) e o ponto \textit{n\_center} (que
está na slice i + 1).\\
Para calcular as coordenadas do ponto que está na superfície do torus precisamos
de calcular um vetor que vai do ponto calculado anteriormente até á
superfície. O raio deste vetor é fixo e é igual a \textit{\_ring\_radius} e o
azimute deste vetor é igual ao \textit{a\_stack} vezes a stack atual mais $\pi$
(para começar de baixo e assim corrigir as texturas). Logo, o único valor que
varia é a insclinação, que é igual a \textit{a\_stack} vezes o indice da stack
atual. Este vetor calculado se for normalizado com recurso ao método
\textit{normalize} é a normal do ponto.\\
Somando para um dado ponto no centro do torus ao vetor correspondente obtemos o
ponto na superfície.\\
A maneira como calculamos as texturas do torus é semelhante à que usamos na
esfera  mas rodada 90º. Assim fica mais fácil criar e editar as texturas.\\
\begin{lstlisting}
float curr_t_x = stack / _stacks;
float curr_t_y = slice / _slices;
float next_t_x = (stack + 1) / _stacks;
float next_t_y = (slice + 1) / _slices;
\end{lstlisting}
Assim, para calcular os \textit{ModelPoint} e criar um vetor (\textit{coords})
com estes utilizamos o seguinte código.

\begin{lstlisting}
for (i32 slice = 0; slice < _slices; slice++) {
    for (i32 stack = 0; stack < _stacks; stack++) {
        // vetores do centro do torus ate a superficie
        float offset = a_stack * stack + M_PI;
        auto base = VectorSpherical(_ring_radius, offset, center.azimuth());
        auto n_base = base.add_azimuth(a_slice);
        auto top = base.add_inclination(a_stack);
        auto n_top = n_base.add_inclination(a_stack);

        // pontos na superficie
        auto p0 = center + base;
        auto p1 = center + top;
        auto p2 = n_center + n_base;
        auto p3 = n_center + n_top;

        // 1st triangle
        coords.emplace_back(p1, top.normalize(), top_t_x, curr_t_y);
        coords.emplace_back(p2, n_base.normalize(), base_t_x, next_t_y);
        coords.emplace_back(p0, base.normalize(), base_t_x, curr_t_y);

        // 2nd triangle
        coords.emplace_back(p1, top.normalize(), top_t_x, curr_t_y);
        coords.emplace_back(p3, n_top.normalize(), top_t_x, next_t_y);
        coords.emplace_back(p2, n_base.normalize(), base_t_x, next_t_y);
    }
}
\end{lstlisting}


\begin{figure}[H]
    \centering
    \begin{minipage}{0.40\textwidth}
        \centering
        \includegraphics[width=\textwidth]{images/donut.jpg}
        \caption{exemplo de textura}
    \end{minipage}\hfill
    \begin{minipage}{0.59\textwidth}
        \centering
        \includegraphics[width=\textwidth]{images/donut_rendered.png}
        \caption{resultado da textura}
    \end{minipage}\hfill
\end{figure}

\section{Cilindro}

\begin{figure}[H]
    \centering
    \begin{minipage}{0.40\textwidth}
        \centering
        \includegraphics[width=\textwidth]{images/cylinder_texture.jpg}
        \caption{exemplo de textura}
    \end{minipage}\hfill
    \begin{minipage}{0.59\textwidth}
        \centering
        \includegraphics[width=\textwidth]{images/cylinder_rendered.png}
        \caption{resultado da textura}
    \end{minipage}\hfill
\end{figure}

\section{Cone}
\begin{figure}[H]
    \centering
    \begin{minipage}{0.40\textwidth}
        \centering
        \includegraphics[width=\textwidth]{images/donut.jpg}
        \caption{exemplo de textura}
    \end{minipage}\hfill
    \begin{minipage}{0.59\textwidth}
        \centering
        \includegraphics[width=\textwidth]{images/donut_rendered.png}
        \caption{resultado da textura}
    \end{minipage}\hfill
\end{figure}

\section{Caixa}
\begin{figure}[H]
    \centering
    \begin{minipage}{0.39\textwidth}
        \centering
        \includegraphics[width=\textwidth]{images/test_cube.jpg}
        \caption{exemplo de textura}
    \end{minipage}\hfill
    \begin{minipage}{0.59\textwidth}
        \centering
        \includegraphics[width=\textwidth]{images/test_cube_rendered.png}
        \caption{resultado da textura}
    \end{minipage}\hfill
\end{figure}

\section{Patches de Bezier}
\begin{figure}[H]
    \centering
    \begin{minipage}{0.40\textwidth}
        \centering
        \includegraphics[width=\textwidth]{images/donut.jpg}
        \caption{exemplo de textura}
    \end{minipage}\hfill
    \begin{minipage}{0.59\textwidth}
        \centering
        \includegraphics[width=\textwidth]{images/donut_rendered.png}
        \caption{resultado da textura}
    \end{minipage}\hfill
\end{figure}

\chapter{Conclusão}
Com este trabalho prático podemos aplicar os conhecimentos leccionados até agora
nas aulas da unidade curricular de computação gráfica permitindo assim
aprofundar os nossos conhecimentos de openGl.\\
Como trabalho futuro gostaríamos de acrescentar iluminação e texturas, melhorar
as \textit{scenes} criadas até agora para usarem essas melhorias e acrescentar
mais modos à camera.

\end{document}
