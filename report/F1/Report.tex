\documentclass[a4paper]{report}
\usepackage[utf8]{inputenc}
\usepackage[portuguese]{babel}
\usepackage{hyperref}
\usepackage{a4wide}
\hypersetup{pdftitle={CC - TP01},
pdfauthor={João Teixeira, José Ferreira, Miguel Solino},
colorlinks=true,
urlcolor=blue,
linkcolor=black}
\usepackage{subcaption}
\usepackage[cache=false]{minted}
\usepackage{listings}
\usepackage{booktabs}
\usepackage{multirow}
\usepackage{appendix}
\usepackage{tikz}
\usepackage{authblk}
\usepackage{bashful}
\usepackage{verbatim}
\usepackage{amsmath}
\usepackage{tikz}
\usepackage{tikz,fullpage}
\usepackage{pgfgantt}
\usepackage{amssymb}
\usetikzlibrary{arrows,%
                petri,%
                topaths}%
\usepackage{tkz-berge}
\usetikzlibrary{positioning,automata,decorations.markings}
\AfterEndEnvironment{figure}{\noindent\ignorespaces}
\AfterEndEnvironment{table}{\noindent\ignorespaces}

\definecolor{solarized@base03}{HTML}{002B36}
\definecolor{solarized@base02}{HTML}{073642}
\definecolor{solarized@base01}{HTML}{586e75}
\definecolor{solarized@base00}{HTML}{657b83}
\definecolor{solarized@base0}{HTML}{839496}
\definecolor{solarized@base1}{HTML}{93a1a1}
\definecolor{solarized@base2}{HTML}{EEE8D5}
\definecolor{solarized@base3}{HTML}{FDF6E3}
\definecolor{solarized@yellow}{HTML}{B58900}
\definecolor{solarized@orange}{HTML}{CB4B16}
\definecolor{solarized@red}{HTML}{DC322F}
\definecolor{solarized@magenta}{HTML}{D33682}
\definecolor{solarized@violet}{HTML}{6C71C4}
\definecolor{solarized@blue}{HTML}{268BD2}
\definecolor{solarized@cyan}{HTML}{2AA198}
\definecolor{solarized@green}{HTML}{859900}

\lstset{
  language=Java,
  upquote=true,
  columns=fixed,
  tabsize=4,
  extendedchars=true,
  breaklines=true,
  numbers=left,
  numbersep=5pt,
  rulesepcolor=\color{solarized@base03},
  numberstyle=\tiny\color{solarized@base01},
  basicstyle=\footnotesize\ttfamily,
  keywordstyle=\color{solarized@green},
  stringstyle=\color{solarized@yellow}\ttfamily,
  identifierstyle=\color{solarized@blue},
  commentstyle=\color{solarized@base01},
  emphstyle=\color{solarized@red}
}

\begin{document}

\title{Computação Gráfica\\
\large Fase 1 - Grupo ??}
\author{José Ferreira (A83683) \and João Teixeira (A85504) \and Miguel Solino (A86435)}
\date{\today}

\begin{center}
    \begin{minipage}{0.75\linewidth}
        \centering
        \includegraphics[width=0.4\textwidth]{images/eng.jpeg}\par\vspace{1cm}
        \vspace{1.5cm}
        \href{https://www.uminho.pt/PT}
        {\color{black}{\scshape\LARGE Universidade do Minho}} \par
        \vspace{1cm}
        \href{https://www.di.uminho.pt/}
        {\color{black}{\scshape\Large Departamento de Informática}} \par
        \vspace{1.5cm}
        \maketitle
    \end{minipage}
\end{center}

\tableofcontents

\chapter{Introdução}
Ao longo deste relatório irá ser explicado a metodologia e raciocínio usados
para a realização da primeira fase de computação gráfica.

\chapter{Utils}
\section{Sistemas de Coordenadas}
De forma a facilitar o desenho das formas foram desenvolvidas um conjunto de
classes para representar dois sistemas distintos de coordenadas.\\
O primeiro sistema de coordenadas é o sistema cartesiano. Neste a posição de um
ponto no espaço fica definido com 3 valores: x, y e z.\\
\begin{figure}[H]
    \centering 
    \includegraphics[width=0.25\textwidth]{images/Coord_XYZ.png}  
    \caption{Coordenadas cartesianas}
    \label{fig:coord_xyz}
\end{figure}
O segundo sistema de coordenadas é o sistema esférico. Neste, para definir a
posição de um ponto no espaço, são também necessários 3 valores: raio,
inclinação e azimute. O raio é a distancia que o ponto tem à origem do
referancial, a inclinação é o angulo que o ponto faz com o eixo vertical e o
azimute é o angulo que o ponto faz com o eixo dos z.
\begin{figure}[H]
    \centering 
    \includegraphics[width=0.3\textwidth]{images/Spherical_Coordinates.png}  
    \caption{Coordenadas esféricas}
    \label{fig:spherical_coordinates}
\end{figure}
Para ambos os sistemas de coordenadas foram definidos uma classe de pontos e uma
classe de vetores.

\chapter{Generator}
O formato escolhido para representar os pontos carregados é o de colocar um
ponto por linha com as coordenadas pela ordem X, Y e Z e separados por espaço.\\
Assim, por exemplo, o conjunto de pontos (0, 0, 0), (1, 0, 0) e (1, 1, 1)
ficaria representado como.
\begin{lstlisting}
0 0 0
1 0 0
1 1 1
\end{lstlisting}

\section{Plano}
Para desenhar um plano com dois triangulos sabendo o lado do plano basta
conseguir obter as coordenadas dos cantos do plano.\\
\begin{figure}[H]
    \centering 
    \includegraphics[width=0.5\textwidth]{images/esquema_plano.png}  
    \caption{Esquema dos pontos no plano}
    \label{fig:scheme_plane}
\end{figure}
Para desenhar um triângulo com o lado visível virado para cima é necessário
seguir a regra da mão direita quando se decide porque ordem desenhar os pontos.
Assim, os dois triângulos necessários são definidos da seguinte forma:\\
Para definir o triângulo superior:
\[(-h/2,\quad 0, \quad -h/2)\]
\[(-h/2,\quad  0, \quad h/2)\]
\[(h/2,\quad  0, \quad -h/2)\]
Para definir o triângulo inferior:
\[(h/2, \quad 0, \quad -h/2)\]
\[(-h/2, \quad  0, \quad  h/2)\]
\[(h/2, \quad  0, \quad h/2)\]

\begin{figure}[H]
    \centering 
    \includegraphics[width=0.5\textwidth]{images/plane.png}  
    \caption{Plano Side:10}
    \label{fig:plane}
\end{figure}


\section{Caixa}
Para definir cada uma das faces de uma caixa é necessário ter um ponto que
esteja nessa face. De forma a minimizar os pontos que são definidos escolhemos
dois pontos em cantos opostos. Neste caso no canto inferior esquerdo da frente e
no canto superior direito de trás. Assim, todas as faces da caixa podem ser
definidas com base em apenas dois pontos.\\
\begin{figure}[H]
    \centering 
    \includegraphics[width=0.5\textwidth]{images/cubo_vectors.png}  
    \caption{Vetores e pontos na caixa}
    \label{fig:box_vectors}
\end{figure}
Para facilitar as contas, o gerador converte as divisões que recebe no númeor de
slices que o cubo tem (\textit{slices = divisions + 1}).\\
Em seguida são definidos 3 vetores cada um associado a um dos planos que
atravessam os eixos e têm como coordenadas o comprimento de cada uma das slices
do cubo.\\
\begin{lstlisting}
float stepx = _x / _slices;
float stepy = _y / _slices;
float stepz = _z / _slices;

Vector xy = Vector(stepx, stepy,     0);
Vector yz = Vector(0    , stepy, stepz);
Vector xz = Vector(stepx,     0, stepz);
\end{lstlisting}
Para obter os pontos ao longo de cada face utilizamos o produto de
\textit{haramard} sobre os vetores definidos anteriormente. Esta operação é
definida para duas matrizes com as mesmas dimensões e produz uma matriz onde
cada elemento (i, j) é o produto de elementos (i , j) das duas matrizes
originais. isto permite escalar os vetores apenas num eixo.\\

\begin{figure}[H]
    \centering 
    \includegraphics[width=0.5\textwidth]{images/box.png}  
    \caption{Caixa: X:5 Y:5 Z:5 Divisions:2}
    \label{fig:box_render}
\end{figure}

\section{Esfera}
Para desenhar a esfera fizemos uso do sistema de coordenadas esféricas definidas
anteriormente. Como o próprio nome indica, o espaço definido por estas
coordenadas é esférico o que facilita o desenho de uma esfera.\\
Tal como foi referido, um ponto precisa de três valores para ser defenido num
sistema de coordenadas esféricas. O primeiro é a dsitância ao ponto do centro do
referencial que por definição também é o raio da esfera. Assim basta obter a
inclinação e o azimute.\\
Para facilitar estas contas são calcolados dois valores:\\
O angulo entre cada slice (ang\_slice = 2 * $\pi$ / n de slices) e o angulo
entre cada stack (ang\_stack = $\pi$ / n de stacks).\\
Assim, para obter todos os pontos numa esfera podemos usar o excerto de código:

\begin{lstlisting}
for(i32 slice = 0; slice < _slices; slice++)
    for(i32 stack = 0; stack < _stacks; stack++)
        coords.push_back(
            PointSpherical(_radius, ang_stack * stack, ang_slice * slice));
\end{lstlisting}

Para obter os próximos pontos da esfera de forma a criar os triangulos basta
fazer \textit{ang\_stack * (stack + 1)} e \textit{ang\_slice * (slice + 1)}\\
Assim, juntando estes quatro pontos com o explicado na criação do plano:

\input{code/spherev2.tex}
O resultado deste código já gera uma esfera. No entanto, quando es está a
desenhar a primeira e a última stack os triangulos são duplicados.
Para resolver este problema basta não desenhar um dos triangulos nestes casos.
Assim a versão final fica:

\begin{lstlisting}
for(i32 slice = 0; slice < _slices; slice++)
    for(i32 stack = 0; stack < _stacks; stack++) {
        if(! (stack == 0 || stack == stacks-1)) {
            //1st triangle
            coords.push_back(
                PointSpherical(_radius, ang_stack * (stack+1), ang_slice * (slice+1)));
            coords.push_back(
                PointSpherical(_radius, ang_stack * stack, ang_slice * (slice+1)));
            coords.push_back(
                PointSpherical(_radius, ang_stack * stack, ang_slice * slice));
        }
        //2nd triangle
        coords.push_back(
            PointSpherical(_radius, ang_stack * (stack+1), ang_slice * slice));
        coords.push_back(
            PointSpherical(_radius, ang_stack * (stack+1), ang_slice * (slice+1)));
        coords.push_back(
            PointSpherical(_radius, ang_stack * stack, ang_slice * slice));
    }
\end{lstlisting}


\section{Cone}

\section{Cilindro}
Para desenhar o cilindro primeiro é definido um vetor que vai da base do cone ao
topo do cone(top) e um vetor que tem a altura de uma stack do cilindro(step) e
é calculado o angulo que uma slice tem(ang\_slice).\\
Assim, para um ponto da aresta da base é dado por \textit{(radius, $\pi$/2,
ang\_slice * slice)}.
\begin{lstlisting}
Vector top = Vector(0, _height, 0);
Vector step = top / _stacks;
float ang_slice = 2 * M_PI / _slices;
\end{lstlisting}
\begin{figure}[H]
    \centering 
    \includegraphics[width=0.5\textwidth]{images/cilindro_vetores.png}  
    \caption{Vetores e angulos dos cilindros}
    \label{fig:cilindro_vec}
\end{figure}
Assim para desenhar a base do cilindro e o topo do cilindro fazemos:
\begin{lstlisting}
for(i32 slice = 0; slice < _slices; slice++) {
    PointSpherical central = PointSpherical(0, 0, 0);
    PointSpherical base    = PointSpherical(_radius, M_PI/2, ang_slice * slice);
    PointSpherical n_base  = PointSpherical(_radius, M_PI/2, ang_slice *(slice+1));
    //top & bottom
    coords.push_back(n_base);
    coords.push_back(base);
    coords.push_back(central);
    coords.push_back(central + top);
    coords.push_back(base + top);
    coords.push_back(n_base + top);
}
\end{lstlisting}


Para desenhar as laterais do cilindro coloca-se um ciclo for dentro do
ciclo definido anteriormente de forma a pecorrer as stacks do sólido.
Assim, a versão final do código fica:

\begin{lstlisting}
for(i32 slice = 0; slice < _slices; slice++) {
    PointSpherical central = PointSpherical(0, 0, 0);
    PointSpherical base    = PointSpherical(_radius, M_PI/2, ang_slice * slice);
    PointSpherical n_base  = PointSpherical(_radius, M_PI/2, ang_slice *(slice+1));
    //top & bottom
    coords.push_back(n_base);
    coords.push_back(base);
    coords.push_back(central);
    coords.push_back(central + top);
    coords.push_back(base + top);
    coords.push_back(n_base + top);

    //side
    for(i32 stack = 0; stack < _stacks; stack++) {
        Point pivot   =   base + step * stack;
        Point n_pivot = n_base + step * stack;

        coords.push_back(pivot);
        coords.push_back(n_pivot + step);
        coords.push_back(pivot + step);
        coords.push_back(pivot);
        coords.push_back(n_pivot);
        coords.push_back(n_pivot + step);
    }
}
\end{lstlisting}



\begin{figure}[H]
    \centering 
    \includegraphics[width=0.5\textwidth]{images/cylinder.png}  
    \caption{Cilindro: altura:5 raio:5 slices:10 stacks:2}
    \label{fig:cilindro_render}
\end{figure}

\chapter{Engine}
Como ainda não existem sombras e luz as formas não são facilmente
distinguíveis. De forma a resolver este problema decidimos atribuir a cada
triângulo uma cor aleatória.
\begin{figure}[H]
    \centering 
    \includegraphics[width=0.5\textwidth]{images/sphere_onecolor.png}  
    \caption{Esfera com cor sólida}
    \label{fig:sphere_color}
\end{figure}
\begin{figure}[H]
    \centering 
    \includegraphics[width=0.5\textwidth]{images/sphere_color.png}  
    \caption{Esfera com várias cores por triângulo}
    \label{fig:sphere_onecolor}
\end{figure}

\section{3D Model loading}
Para facilitar a leitura do ficheiro final gerado pelo gerador foi desenvolvida
uma classe denominada de \textit{Model}. Esta contem um vetor de pontos que são
carregados do ficheiro anteriormente gerado e para o povoar apenas é necessário
indicar o nome do ficheiro.\\
\begin{lstlisting}
Model::Model(string fileName){
    float x, y, z;
    ifstream file(fileName.c_str());
    while (file >> x >> y >> z)
         points.push_back(Point(x, y, z));
}
\end{lstlisting}
Esta classe também contém um método para desenhar o modelo \textit{draw\_model} que chama um método
para desenhar triângulos com cores aleatórias \textit{draw\_triangle}.
\begin{lstlisting}
void draw_triangle(Point p1, Point p2, Point p3){
    float r = static_cast <float> (rand()) / static_cast <float> (RAND_MAX);
    float g = static_cast <float> (rand()) / static_cast <float> (RAND_MAX);
    float b = static_cast <float> (rand()) / static_cast <float> (RAND_MAX);
    glColor3f(r, g, b);
    glBegin(GL_TRIANGLES);
        glVertex3f(p1.x(), p1.y(), p1.z());
        glVertex3f(p2.x(), p2.y(), p2.z());
        glVertex3f(p3.x(), p3.y(), p3.z());
    glEnd();
}

void Model::draw_model() {
    for(i32 i = 0; i < points.size(); i += 3)
        draw_triangle(points[i], points[i+1], points[i+2]);
}
\end{lstlisting}

\section{XML parsing}
Para dar parsing ao XML utilizamos a biblioteca \textit{rapidxml}. Nesta
primeira versão do parser apenas obtemos quais são os ficehiros dos modelos a
carregar contidos no xml. De forma a garantir que o ficheiro é lido apenas uma
vez o carregamento é feito imediatamente quando o programa inicia e os ficheiros
são carregas para a classe \textit{Models} que contem um vetor de \textit{Model}
e o resultado é guardado num singleton.\\
Assim, para carregar todos os sólidos dentro do xml basta fazer:
\begin{lstlisting}
models = Models(read_models(sceneName));
\end{lstlisting}
Para desenhar os modelos utilizamos o método \textit{draw\_models()}.

\section{Camera}
A camera consiste em dois pontos. Um ponto de coordenadas esféricas que
representa a localização da camera no espaço e um ponto de coordenadas
cartesianas que representa para onde a camera está a olhar.\\
\begin{lstlisting}
PointSpherical _pl;
Point _center;
\end{lstlisting}
Para desenhar a camera converte se o ponto esférico num ponto cartesiano e
utiliza-se a função \textit{gluLookAt}.
\begin{lstlisting}
Point c = Point(_pl);
glLoadIdentity();
gluLookAt(c.x(),c.y(),c.z(), 
	_center.x(),_center.y(),_center.z(),
		0.0f,1.0f,0.0f);
\end{lstlisting}
Para mover a camera basta utilizar as funções definidas na classe \textit{Point}
e \textit{PointSpherical} para alterar a posição da camera.



\end{document}
